\documentclass{article}
\usepackage[utf8]{inputenc}
\usepackage{amsmath}
\usepackage[ruled,vlined]{algorithm2e}
\usepackage{graphicx}
\usepackage{float}


\title{Segmentation 3 (Water shed algorithm)}
\author{Cristian Manuel Abrante Dorta}
\date{April 2020}

\begin{document}

\maketitle

\section{Pseudo code}

In this section we are going to present the pseudo code of the implementation of the algorithm based on \cite{watershed}.First, we are going to present the procedure for arrowing the pixels of the image: \\

\begin{algorithm}[H]
 \KwData{and input image $I$}
 \KwResult{A list of pixels sets $W = <P_1, \dots, P_n>$\;}
 \emph{D is a matrix of directions initialized to Null.}\\
 localMinimums = []
 
 \For{$p \in I$}{
    \If{$D(p) \neq Null$}{
        \tcp{Auxiliary method for finding best neighbour (less gray level).}\label{cmt}
        bestNeighbour = findBestNeighbour(I, p)
        
        \If{bestNeighbour = Null}{
            \tcp{In this case best neighbour is not found so we have regional minimum.}\label{cmt}
            localMinimum := p\\
            $D(p) = CENTER$
        }
        \If{$I(bestNeighbour) = I(p)$}{
            \tcp{In this case we should explore the plateau.}\label{cmt}
            explorePlateau(p, D)
        }
        \If{$I(bestNeighbour) < I (p)$}{
            \tcp{Normal pixel, we have to calculate direction.}\label{cmt}
            direction = computeDirection(p, bestNeighbour);\\
            D(p) = direction.
        }
    }
 }
 \tcp{Finally the watersheds have to be constructed from the directions.}\label{cmt}
 W = constructWatersheds(I, D);
 \caption{Arrowing method}
\end{algorithm}

\vspace{}

After that, we are going to explain the procedure of the exploration of the plateau which is the novel method of this algorithm. \\

\begin{algorithm}[H]
 \KwData{and input image $I$ and an initial pixel $p$}
 \emph{$Q$ is a queue}\\
 \emph{$P$ is the plateau}\\
 \emph{$bestNeighbour$ is the best neighbour of the plateau}\\
 
 Q.add(p)\\
 p := p

 \While{Q.empty() $\neq$ false}{
   currentP = Q.pop()\\
   \tcp{Auxiliary method for finding neighbours.}\label{cmt}
   \For{$q \in Neighbours(I, currentP)$}{
     \If{$q$ is not visited}{
        mark $q$ as visited\\
        \If{$I(q) < I(p) \wedge I(q) < I(bestNeighbour)$}{
            \tcp{Best neighbour found}\label{cmt}
            bestNeighbour = q
        }
        \If{$I(q) = I(p)$}{
            Q.push(q)\\
            P := q
        }
     }
   }
 }
 
 \eIf{$bestNeighbour \neq Null$}{
    \emph{plateau is not minimal}\\
 }{
    \emph{plateau is minimal}\\
 }

 \caption{Plateau exploration method}
\end{algorithm}

\section{Example}

For testing if the implementation of the algorithm is correct (implementation can be found in the source code), I built an squared image of 10 x 10 pixels, and we can see here the result after applying the segmentation algorithm:

\begin{figure}[!h]
\minipage{0.5\textwidth}
    \centering
  \includegraphics[scale=5]{report/segmentation/image/test.png}
  \caption{Original image}\label{fig:image-1}
\endminipage\hfill
\minipage{0.5\textwidth}
    \centering
  \includegraphics[scale=5]{report/segmentation/image/test-out.png}
  \caption{Watershed regions obtained}\label{fig:awesome_image2}
\endminipage\hfill
\end{figure}

\begin{thebibliography}{9}
\bibitem{watershed}
   Suphalakshmi, A and Anandhakumar, P
    \newblock {\em An improved fast watershed algorithm based on finding the shortest paths with breadth first search}, 2012.

\end{thebibliography}

\end{document}

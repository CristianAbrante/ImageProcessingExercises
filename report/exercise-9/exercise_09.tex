\documentclass{article}
\usepackage[utf8]{inputenc}
\usepackage{amsmath}

\title{Exercise 09}
\author{Cristian Manuel Abrante Dorta}
\date{March 2020}

\begin{document}

\maketitle

Proof the idempotence of the opening ($\gamma$) closing ($\varphi$) operation.\\

In a formal way, it can be expressed as:

\begin{equation}
    (\varphi \gamma)(\varphi \gamma) (I) = \varphi \gamma(I) 
\end{equation}

It is equivalent to prove the following:

\begin{equation}
    (\varphi \gamma)(\varphi \gamma) (I) \leq \varphi \gamma(I) \land 
    (\varphi \gamma)(\varphi \gamma) (I) \geq \varphi \gamma(I)
    \rightarrow (\varphi \gamma)(\varphi \gamma) (I) = \varphi \gamma(I) 
\end{equation}

\section{Proof: $(\varphi \gamma)(\varphi \gamma) (I) \leq \varphi \gamma(I)$}

First we are going to construct a proof for $(\varphi \gamma)(\varphi \gamma) (I) \leq \varphi \gamma(I)$. For doing this, as the closing is an idempotent operation we can consider that:

\begin{equation}
\begin{split}
    \varphi \varphi \varphi \gamma(I) = \varphi \gamma(I)
\end{split}
\end{equation}

As the opening is antiextensive and closing is extensive, we can affirm that the closing is greater or equal than the opening ($\varphi \geq \gamma$). Substituting the second closing operation by an opening we can confirm that:

\begin{equation}
    \varphi \gamma \varphi \gamma(I) \leq \varphi \gamma(I)
\end{equation}

\section{Proof: $(\varphi \gamma)(\varphi \gamma) (I) \geq \varphi \gamma(I)$}

Second, we are going to prove that $(\varphi \gamma)(\varphi \gamma) (I) \geq \varphi \gamma(I)$. For doing this, as the opening is idempotent we can assume that:

\begin{equation}
\begin{split}
    \varphi \gamma \gamma \gamma(I) = \varphi \gamma(I) \\
\end{split}
\end{equation}

As the opening is antiextensive, and closing is extensive. We can affirm that the opening is greater or equal than the opening ($\varphi \geq \gamma$). Substituting the second opening operation by a closing we can affirm that:

\begin{equation}
     \varphi \gamma \varphi \gamma(I) \geq \varphi \gamma(I) \\
\end{equation}

\section{Conclusions}

Finally, as the two parts of the and ($\land$) are proved, we have proved the equality of the both assumptions. So we can affirm that the opening-closing operation is idempotent.

\end{document}
